\documentclass[UTF8]{ctexart}
\title{biochem}
\author{wengang-yin}
\date{2020-11-09}
\bibliographystyle{plain}

\begin{document}
\maketitle
\tableofcontents
\section{核酸}
\subsection{核苷与核苷酸}
  \begin{enumerate}
  \item 命名:
    \begin{itemize}
      \item 次黄苷\ I\ insine
      \item 黄嘌呤核苷\ X\ xanthosine
      \item 二氢尿嘧啶核苷\ D\ dihydroiridine
      \item 假尿嘧啶核苷\ $\psi$\  pseudouridine
    \end{itemize}
    \par 取代集团用小写字母表示,碱基取代在左,核糖取代在右.取代位置标在取代集团右上角,
    取代个数右下.例如5-甲基脱氧胞苷的符号为$m^5dC$.
    \item 结构
    \par 核苷酸是核苷的磷酸酯.核苷酸中的核糖有三个自由的羟基,可均被磷酸酯化分别形成
    $2'-$,$3'-$和$5'-$核苷酸.而脱氧核苷酸的戊糖上只有两个自由羟基,只能生成$3$和$5$位
    的脱氧核苷酸.
  \end{enumerate}
  \subsection{核酸的一级结构}
  \par DNA和RNA都是没有分支的多核苷酸长链,链中每个核苷酸的\emph{\large $3',5'-$磷酸二酯键}.
  \par 相间排列的戊糖和磷酸构成核酸大分子的主链,而奸计则可以看作有次序连接在主链上的
  侧链基团.每一个线性核酸链都有一个$3'$和$5'$端.
  \subsection{DNA的二级结构}
  \begin{enumerate}
    \item DNA分子由两条相反方向的平行多核苷酸链构成,一条链的$5'$端与另一条链的$3'$端相
    对,两条链沿共同的螺旋轴成右手螺旋.
    \item 两条链上的碱基均在主链内侧,A-T\ C-G配对.A与T之间形成两条氢键,G与C配对形成三条氢键.
    双螺旋DNA分子的螺旋直径为$2nm$.
  \end{enumerate}
\end{document}
